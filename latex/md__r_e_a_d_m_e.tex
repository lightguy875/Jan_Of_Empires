Simple Strategy game

\section*{Ideia do jogo}

É um jogo de estratégia em turnos onde você controla 3 unidades para explorar, adquirir recursos, aumentar o poder da sua unidade usando os pilares para vencer o oponente. Cada tipo de unidade tem vantagem sobre outra unidade. O jogo é single-\/player com 1 C\+PU de oponente.

\section*{Elementos do Jogo}

\subsection*{Recursos}

Os recursos servem para criar pilares novos, criar necromancers (a partir de pilares) ou aumentar o poder de necromancers

Dois tipos de recursos\+:
\begin{DoxyItemize}
\item \mbox{\hyperlink{class_metal}{Metal}}\+: para criação de pilares. Menos abundante no mapa (terá o suficiente para criar mais 2 pilares para cada jogador).
\item \mbox{\hyperlink{class_ossos}{Ossos}}\+: para aumento de poder das unidades. Mais abundante no mapa.
\end{DoxyItemize}

\subsection*{Prédios (Pilares)}

Os prédios são pilares. Cada pilar representa uma fábrica de um tipo de unidade que será criada a partir da quantidade de ossos. Cada pilar cria uma única unidade, que é o necromancer. Depois de criado o necromancer, o pilar pode aumentar o poder do necromancer utilizando ossos.

Cada pilar tem\+:
\begin{DoxyItemize}
\item HP\+: representa a vida do pilar. Se chega a zero, ele é destruído.
\end{DoxyItemize}

É possível reconstruir pilares, com o máximo de 3 por jogador.

Três tipos de pilares\+:
\begin{DoxyItemize}
\item \mbox{\hyperlink{class_pilar}{Pilar}} da Espada\+: é o pilar que cria o necromancer guerreiro.
\item \mbox{\hyperlink{class_pilar}{Pilar}} da Lança\+: é o pilar que cria o necromancer cavaleiro.
\item \mbox{\hyperlink{class_pilar}{Pilar}} do Arco \+: é o pilar que cria o necromancer arqueiro.
\end{DoxyItemize}

\subsection*{Unidades (Necromancers)}

As unidades são representadas por um único elemento, o \mbox{\hyperlink{class_necromancer}{Necromancer}}. Cada necromancer tem um número associado que define quanto de poder ele tem. Por exemplo\+: no mapa há uma unidade A com o número 20. Isso significa que a unidade A tem 20 de poder.

Cada unidade tem\+:
\begin{DoxyItemize}
\item MP\+: representa quanto de poder a unidade tem e representa a vida da unidade. Se ela chegar a zero, a unidade morre.
\end{DoxyItemize}

São 3 tipos de unidades\+:


\begin{DoxyItemize}
\item \mbox{\hyperlink{class_necromancer}{Necromancer}} \mbox{\hyperlink{class_guerreiro}{Guerreiro}} (A)\+: é o necromancer que invoca somente undeads guerreiros. A quantidade de undeads é o que define o poder do necromancer (é o número que mostra a força da unidade). Ele tem vantagem sobre o \mbox{\hyperlink{class_necromancer}{Necromancer}} \mbox{\hyperlink{class_cavaleiro}{Cavaleiro}} e desvantagem sobre o \mbox{\hyperlink{class_necromancer}{Necromancer}} \mbox{\hyperlink{class_arqueiro}{Arqueiro}}.
\item \mbox{\hyperlink{class_necromancer}{Necromancer}} \mbox{\hyperlink{class_cavaleiro}{Cavaleiro}} (B)\+: é o necromancer que invoca somente undeads cavaleiros. A quantidade de undeads é o que define o poder do necromancer (é o número que mostra a força da unidade). Ele tem vantagem sobre o \mbox{\hyperlink{class_necromancer}{Necromancer}} \mbox{\hyperlink{class_arqueiro}{Arqueiro}} e desvantagem sobre o \mbox{\hyperlink{class_necromancer}{Necromancer}} \mbox{\hyperlink{class_guerreiro}{Guerreiro}}.
\item \mbox{\hyperlink{class_necromancer}{Necromancer}} \mbox{\hyperlink{class_arqueiro}{Arqueiro}} (C)\+: é o necromancer que invoca somente undeads arqueiros. EA quantidade de undeads é o que define o poder do necromancer (é o número que mostra a força da unidade). Ele tem vantagem sobre o \mbox{\hyperlink{class_necromancer}{Necromancer}} \mbox{\hyperlink{class_guerreiro}{Guerreiro}} e desvantagem sobre o \mbox{\hyperlink{class_necromancer}{Necromancer}} \mbox{\hyperlink{class_cavaleiro}{Cavaleiro}}.
\end{DoxyItemize}

O máximo de unidades no jogo são 6.

\section*{Regras definidas para o jogo}

\subsection*{Inicialização}

Inicialmente cada jogardor possui\+:
\begin{DoxyItemize}
\item 1 \mbox{\hyperlink{class_pilar}{Pilar}} da Espada
\item 1 \mbox{\hyperlink{class_necromancer}{Necromancer}} guerreiro
\end{DoxyItemize}

\subsection*{Turno}

Em UM turno o jogador pode criar pilares, fortalecer unidades e mover apenas uma delas. Ele acaba quando o jogador move uma peça ou decide terminar o turno.

\subsection*{Captação de recursos}

Os recursos são espalhados randômicamente no mapa no início do jogo. São capturados quando o jogador move uma unidade para o mesmo bloco em que o recurso está inserido.

\subsection*{Combate}

O combate acontece automaticamente quando duas unidades estão vizinhas O combate acontece sempre entre apenas duas unidades No combate, verificam-\/se os poderes de cada unidade, e as capacidades e fraquezas de cada uma em relação a outra, de acordo com os seus tipos. Então, é decrescido poder de cada uma de acordo com o estabelecido.

\subsection*{Condições de término do jogo}

As condições para o término do jogo são os seguintes\+:
\begin{DoxyItemize}
\item Se todos os pilares do oponente forem destruídos;
\item Se todos os recursos do mapa acabarem\+: nesse caso, a quantidade da vida dos pilares + unidades é somado. Quem tiver mais, vence.
\end{DoxyItemize}

\section*{Useful links}


\begin{DoxyItemize}
\item \href{http://lazyfoo.net/tutorials/SDL/}{\tt S\+DL Tutorial link}
\item \href{https://www.libsdl.org/release/SDL-1.2.15/docs/html/}{\tt Bibliotecas de S\+DL} 
\end{DoxyItemize}